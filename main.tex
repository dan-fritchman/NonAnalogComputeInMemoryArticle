% 
% # Main Article Latex 
% 
% Solely sets up packages, formatting, meta-data, and the like. 
% All content is offloaded to markdown-package-read `article.md`. 
% 

% Header content from IEEE
\documentclass[twoside,9pt,journal,letterpage]{IEEEtran}
\usepackage{cite}
\usepackage{amsmath,amssymb,amsfonts}
\usepackage{algorithmic}
\usepackage{graphicx}
\usepackage{textcomp}
\usepackage{placeins}
\usepackage{tabularx}
\usepackage{svg}
\def\BibTeX{{\rm B\kern-.05em{\sc i\kern-.025em b}\kern-.08em
    T\kern-.1667em\lower.7ex\hbox{E}\kern-.125emX}}

% Header content from Overleaf's Markdown example
\usepackage[fencedCode,citations,definitionLists,hashEnumerators,smartEllipses,pipeTables,tableCaptions,hybrid]{markdown}
\usepackage[utf8]{inputenc}
\usepackage[T1]{fontenc}
\usepackage[a4paper,margin=1.5cm]{geometry}
\usepackage{fullpage}
\usepackage[numbers]{natbib}
\usepackage{minted}

\setkeys{Gin}{width=\linewidth,totalheight=\textheight,keepaspectratio}

%% You can re-define how links are rendered: uncomment the following to get hyperlinked text instead of footnotes

\usepackage{hyperref}
%%begin novalidate
\markdownSetup{rendererPrototypes={
  link = {\href{#2}{#1}},
  image = {\begin{figure}[hbt!]
    \centering
    \includegraphics{#3}%
    \ifx\empty#4\empty\else
    \caption{#4}%
    \fi
    \label{fig:#1}%
    \end{figure}}
}}
%%end novalidate

\renewcommand{\tabularxcolumn}[1]{m{#1}}

\let\labelindent\relax
\usepackage[shortlabels]{enumitem}
\usepackage{float}
\setlength{\textfloatsep}{5pt}

\usepackage{datetime}
\newdateformat{monthdayyeardate}{%
  \monthname[\THEMONTH]~\THEDAY, \THEYEAR}

\newcommand{\titlestr}{All-Digital In-Memory Computation for Machine Learning and Neural Network Acceleration}
\title{\titlestr}

\usepackage{lipsum}

\author{
	Dan Fritchman, \IEEEmembership{Member, IEEE} 
	\thanks{Date of publication \monthdayyeardate\today.}
	\thanks{D. Fritchman is with the Department of Electrical Engineering and Computer Sciences, University of California, Berkeley, Berkeley, CA 94720 USA (e-mail: dan\_fritchman@berkeley.edu).}
	\thanks{This work was supported by Professor Sophia Shao.}
}
% \date{}
% \markboth{UC Berkeley Proceedings of EE290-2, May 2021}{Fritchman, \titlestr}
% \IEEEpubid{~\copyright~2020 University of California, Berkeley}

\clearpage
\begin{document}
\maketitle
\IEEEpeerreviewmaketitle

\begin{markdown}

% From the Overleaf examples:
% Including an external \texttt{.md} file, distributed with the \texttt{markdown} package:
% \markdownInput{example.md}
% \markdownInput{more.md}

% Our actual content:  

\markdownInput{article.md}

\end{markdown}

% Sadly tables must be in Latex. 

\begin{table}[!ht]
\caption{Power, Performance, and Area of CIM and Gemmini Systolic Arrays in 28nm CMOS} 
\label{table:comparison}
\centering
\begin{tabularx}{\columnwidth}{
	| >{\centering\arraybackslash}X 
	| >{\centering\arraybackslash}X 
	| >{\centering\arraybackslash}X 
	| >{\centering\arraybackslash}X 
	| >{\centering\arraybackslash}X 
	| >{\centering\arraybackslash}X 
	| >{\centering\arraybackslash}X | }
	\hline
	& Dims & Area (µm2) & Freq (GHz) & Power (mW) & Ops/s & Ops/W \\
	\hline
	CIM Array & 8b 128x128 & 490k & 0.4 & 98.2 & 1638G & 16.7T \\
	\hline
	Gemmini (Systolic) Array & 8b 16x16 & 518k & 1.0 & 3177 & 512G & 161G \\
	\hline
\end{tabularx}
\end{table}

\bibliographystyle{ieeetr}
\bibliography{refs}
\end{document}